\section{Results}
We experimented with many different configurations of classifiers and pre-processing. The results did not differ much, so we decided to include 4 classifier in their default configurations (Naive Bayes, Decision Tree - C4.5, SVM, and Bayesian Networks) and a boosted C4.5 Decision Tree using AdaBoost.

Table 1 shows the results of the 10-fold cross-validation. We can see that SVM and Decision Trees achieved the best accuracies which, after performing a corrected paired t-test, are better than Naive Bayes and Bayesian Networks at the level of significance 0.01.

Naive Bayes, despite having the lowest accuracy, was the fastest among all algorithms and has the lowest false positive rate (malware applications predicted as normal ones), i.e. classified the majority of malware applications correctly. On the other hand, its false positive rate, after a corrected paired t-test, is only significantly better than Bayesian Networks. Overall, SVM seems to be the best classifier: it has the best accuracy, a low false negative rate and a relatively low execution time.
\captionof{table}{Overall Comparison}
\begin{tabular}{|l|l l l l|}
\hline
Method & False Positive & False Negative & Accuracy & Training + Validation Time \\
\hline
Naive Bayes & 8.5\% & 7.8\% & 92.1\% & 1.2s \\
Decision Tree & 12.4\% & 0.8\% & 97.4\% & 6.6s \\
Bayes Net & 22.6\% & 0.6\% & 96.0\% & 1.5s \\
SVM (SMO) & 9.2\% & 0.5\% & 98.2\% & 6.9s \\
AdaBoost (C4.5) & 9.5\% & 0.1\% & 98.4\% & 68.3s \\
\hline
\end{tabular}
\captionof{table}{Naive Bayes: Confusion Matrix}
\begin{tabular}{|l|l l|}
\hline
Actual / Prediction & Malware & Normal \\
\hline
Malware & 259 & 24 \\
Normal & 121 & 1436 \\
\hline
\end{tabular}
\captionof{table}{Decision Tree: Confusion Matrix}
\begin{tabular}{|l|l l|}
\hline
Actual / Prediction & Malware & Normal \\
\hline
Malware & 221 & 62 \\
Normal & 11 & 1546 \\
\hline
\end{tabular}
\captionof{table}{Bayesian Networks: Confusion Matrix}
\begin{tabular}{|l|l l|}
\hline
Actual / Prediction & Malware & Normal \\
\hline
Malware & 221 & 66 \\
Normal & 9 & 1548 \\
\hline
\end{tabular}
\captionof{table}{SVM (SMO): Confusion Matrix}
\begin{tabular}{|l|l l|}
\hline
Actual / Prediction & Malware & Normal \\
\hline
Malware & 248 & 35 \\
Normal & 12 & 1545 \\
\hline
\end{tabular}
\captionof{table}{AdaBoost (C4.5): Confusion Matrix}
\begin{tabular}{|l|l l|}
\hline
Actual / Prediction & Malware & Normal \\
\hline
Malware & 225 & 58 \\
Normal & 30 & 1527 \\
\hline
\end{tabular}


\subsection{Best Features}
To get some insight into what features were most useful, we ranked features by the Information Gain Ratio and the top 20 ones are listed in the table below. Some features may be correlated, such as using an API call sendTextMessage and requiring a SEND\_SMS permission.

As we can see, the most useful features contain the usage of telephony package and relevant API calls, reading the phone's status, various String and Byte operations, networking, or the permission to install new applications. 

\begin{tabular}{l l}
 Feature & Information Gain Ratio \\
 TELEPHONY & 0.317 \\
 sendTextMessage & 0.308 \\
 DEXLEN & 0.264 \\
 android.permission.SEND\_SMS & 0.245 \\
 append & 0.233 \\
 android.permission.READ\_PHONE\_STATE & 0.229 \\
 getDeviceId & 0.221 \\
 valueOf & 0.207 \\
 METHODS & 0.205 \\
 setRequestMethod & 0.203 \\
 write & 0.192 \\
 CLASSES & 0.189 \\
 read & 0.188 \\
 getNetworkOperator & 0.186 \\
 indexOf & 0.185 \\
 substring & 0.181 \\
 XML & 0.175 \\
 android.permission.INSTALL\_PACKAGES & 0.165 \\
 FILENO & 0.163 \\
 FIELDS & 0.162 \\
\end{tabular}

