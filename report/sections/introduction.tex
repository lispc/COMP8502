\section{Introduction}
Smartphones became a part of our daily lives and apart from the standard phone functionality, such as calls, SMS and MMS messages, we use them in many different scenarios: reading e-mails and replying to them, readings news, navigating on a map, browsing personal and corporate social networks, taking pictures and uploading them to cloud repositories, playing games, checking our bank account balances or even paying for small purchases using NFC. Many of these scenarios deal with our private data that could be possibly misused by a malicious application.

We can add any extra functionality to our smartphone simply by installing new applications, just like on our personal computers. This can be done via the official central repositories, such as Google Play  \cite{google_play} or iOS App Store, which try to verify that applications do not contain a malicious code. The situation is different with Android, the most popular smartphone platform with a 79\% share of the global market \cite{idc}. Android gives users a freedom to choose third party sites instead of Google Play as well as to install a manually downloaded apps. This freedom comes with a price of a greater vulnerability to malicious applications.

In this project, we collected a sample of 1840 Android applications (283 malware ones, and 1557 non-malware), implemented a tool to extract static features of each application, evaluated .... TODO

The rest of this report is organised as follows. Section 2 briefly overviews recent research that inspired this project. Section 3 describes the collected dataset, extracted features, and our experimental approach. Section 4 presents our results. Section 5 discusses limitations of this project and potential improvements. Section 6 summarises the overall report.

\section{Related Work}
TODO - Shabtai \cite{shabtai_2010}

TODO - Batyuk \cite{batyuk_2011}

TODO - Sanz \cite{sanz_2012} and others?